\documentclass[11pt]{article}

\usepackage{amsmath}
\usepackage{amsthm}
\usepackage{amsfonts}
\usepackage{amssymb}
\usepackage[utf8]{inputenc}
\usepackage[T1]{fontenc}
\usepackage[shortlabels]{enumitem}

\usepackage{listings}
\usepackage{color}
%New colors defined below
\definecolor{codegreen}{rgb}{0,0.6,0}
\definecolor{codegray}{rgb}{0.5,0.5,0.5}
\definecolor{codepurple}{rgb}{0.58,0,0.82}
\definecolor{backcolour}{rgb}{0.95,0.95,0.92}

%Code listing style named "mystyle"
\lstdefinestyle{mystyle}{
  backgroundcolor=\color{backcolour},   commentstyle=\color{codegreen},
  keywordstyle=\color{magenta},
  numberstyle=\tiny\color{codegray},
  stringstyle=\color{codepurple},
  basicstyle=\footnotesize,
  breakatwhitespace=false,         
  breaklines=true,                 
  captionpos=b,                    
  keepspaces=true,                 
  numbers=left,                    
  numbersep=5pt,                  
  showspaces=false,                
  showstringspaces=false,
  showtabs=false,                  
  tabsize=2
}

\lstset{style=mystyle}

\topmargin -.5in
\textheight 9in
\oddsidemargin -.25in
\evensidemargin -.25in
\textwidth 7in

\begin{document}

\newtheoremstyle{break}% name
  {}%         Space above, empty = `usual value'
  {}%         Space below
  {}% Body font
  {}%         Indent amount (empty = no indent, \parindent = para indent)
  {\bfseries}% Thm head font
  {.}%        Punctuation after thm head
  {\newline}% Space after thm head: \newline = linebreak
  {}%         Thm head spec

\theoremstyle{definition}
\theoremstyle{break}
\newtheorem{exerc}{Exercício}

\author{Lucas Resende Pellegrinelli Machado (2018126673)}
\title{Algebra Linear Computacional - Lista de Exercicios 3}
\maketitle

\medskip

\begin{exerc}
\begin{enumerate}[a.]
.
\item
O script requerido pelo exercicio é
\begin{lstlisting}[language=Python]
import numpy as np

def func(m):
  W = np.abs(np.random.randn(m, 4))
  W_ = np.sqrt(W)
  Z = W_.T @ W_
  print(Z)
  return np.linalg.norm(Z - np.eye(4))
\end{lstlisting}

\item
A diferença de normas obtida com $m = 100$ é $\approx 0.44$. Já para $m = 10000$, a diferença é $\approx 0.043$.

\item
A matriz se aproxima de uma matriz ortogonal à medida que que $m$ aumenta.

\end{enumerate}
\end{exerc}

\begin{exerc}
\begin{enumerate}[a.]
.
\item
Como a fórmula para calcular projeções entre vetores é

$$Proj_u(x) = \frac{x \cdot u}{||u||} u$$

Temos que no nosso exemplo, usando o vetor $u$

$$Proj_x(u) = \frac{(1, 1, 0) \cdot (1, 2, 3)}{||(1, 1, 0)||} (1, 1, 0)$$

Que desenvolvendo temos:

$$Proj_x(u) = \frac{1 + 2 + 0}{\sqrt{2}} (1, 1, 0) = (\frac{3}{\sqrt{2}}, \frac{3}{\sqrt{2}}, 0)$$

Já usando o vetor $v$, semelhantemente temos

$$Proj_x(v) = \frac{(1, -1, 1) \cdot (1, 2, 3)}{||(1, -1, 1)||} (1, -1, 1)$$

Que desenvolvendo obtemos:

$$Proj_x(u) = \frac{1 - 2 + 3}{\sqrt{3}} (1, -1, 1) = (\frac{2}{\sqrt{3}}, -\frac{2}{\sqrt{3}}, \frac{2}{\sqrt{3}})$$

\item
Como a projeção de um vetor $x$ em um espaço $S$ formado por vetores $v_i$ é dado por:

$$Proj_x(S) = \sum_{v_i \in S} Proj_x(v_i)$$

Temos que:

$$Proj_x({u, v}) = Proj_x(u) + Proj_x(v) = (\frac{3}{\sqrt{2}}, \frac{3}{\sqrt{2}}, 0) + (\frac{2}{\sqrt{3}}, -\frac{2}{\sqrt{3}}, \frac{2}{\sqrt{3}})$$
$$ = (\frac{3}{\sqrt{2}} + \frac{2}{\sqrt{3}}, \frac{3}{\sqrt{2}} - \frac{2}{\sqrt{3}}, \frac{2}{\sqrt{3}}) \neq (1, 2, 3) = x$$

Isso acontece pois só com 2 vetores ($u$ e $v$) não é possível formar um espaço em $R^3$, logo não é possível obter o vetor $x$ (que é em $R^3$) a partir de uma combinação linear entre $u$ e $v$.

\end{enumerate}
\end{exerc}

\begin{exerc}
\begin{enumerate}[a.]
Como a matriz de covariancia é dada por

$$C_x = \frac{1}{n - 1}X^TX$$

depois de ja ter tratado a matriz subtraindo de cada coluna sua média


Podemos fazer um script em python para calcula-la:
\begin{lstlisting}[language=Python]
import numpy as np

M = np.array([
  [90, 80, 60, 95],
  [65, 75, 90, 70],
  [40, 90, 60, 55],
  [80, 60, 59, 75],
  [60, 100, 80, 80]
])

A = M - M.mean(axis=0, keepdims=True)

C = (A.T @ A) / (A.shape[0] - 1)
print(C)
\end{lstlisting}

Que nos da como resposta

$$\begin{bmatrix}
	370 & -165 & -53.25 & 243.75\\
	-165 & 230 & 52 & -18.75\\
	-51 & 52 & 190.5 & -12.5\\
	243.75 & -18.75 & -12.5 & 212.5\\
\end{bmatrix}$$


\end{enumerate}
\end{exerc}

\newpage

\begin{exerc}
\begin{enumerate}[a.]
.
\item
Verdadeiro.

\item
Falso.

\item
Falso.

\item
Verdadeiro

\item
Falso

\item
Falso

\end{enumerate}
\end{exerc}

\end{document}
\grid
\grid