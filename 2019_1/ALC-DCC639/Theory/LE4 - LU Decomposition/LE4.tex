\documentclass[11pt]{article}

\usepackage{amsmath}
\usepackage{amsthm}
\usepackage{amsfonts}
\usepackage{amssymb}
\usepackage[utf8]{inputenc}
\usepackage[T1]{fontenc}
\usepackage[shortlabels]{enumitem}

\usepackage{listings}
\usepackage{color}
%New colors defined below
\definecolor{codegreen}{rgb}{0,0.6,0}
\definecolor{codegray}{rgb}{0.5,0.5,0.5}
\definecolor{codepurple}{rgb}{0.58,0,0.82}
\definecolor{backcolour}{rgb}{0.95,0.95,0.92}

%Code listing style named "mystyle"
\lstdefinestyle{mystyle}{
  backgroundcolor=\color{backcolour},   commentstyle=\color{codegreen},
  keywordstyle=\color{magenta},
  numberstyle=\tiny\color{codegray},
  stringstyle=\color{codepurple},
  basicstyle=\footnotesize,
  breakatwhitespace=false,         
  breaklines=true,                 
  captionpos=b,                    
  keepspaces=true,                 
  numbers=left,                    
  numbersep=5pt,                  
  showspaces=false,                
  showstringspaces=false,
  showtabs=false,                  
  tabsize=2
}

\lstset{style=mystyle}

\topmargin -.5in
\textheight 9in
\oddsidemargin -.25in
\evensidemargin -.25in
\textwidth 7in

\begin{document}

\newtheoremstyle{break}% name
  {}%         Space above, empty = `usual value'
  {}%         Space below
  {}% Body font
  {}%         Indent amount (empty = no indent, \parindent = para indent)
  {\bfseries}% Thm head font
  {.}%        Punctuation after thm head
  {\newline}% Space after thm head: \newline = linebreak
  {}%         Thm head spec

\theoremstyle{definition}
\theoremstyle{break}
\newtheorem{exerc}{Exercício}

\author{Lucas Resende Pellegrinelli Machado (2018126673)}
\title{Algebra Linear Computacional - Lista de Exercicios 4}
\maketitle

\medskip

\begin{exerc}
\begin{enumerate}[a.]
\
\item
Dado que a determinante da matriz é 60 (ou seja $\neq 0$) e que seu posto é $3$ (ou seja, todas as colunas são linearmente independentes) sabemos que o sistema possuí uma única solução.
\item
Dada a matriz da questão, temos também o seguinte sistema:

$$\begin{cases}
4x_1 + 0x_2 + 0x_3 = 12\\
-2x_1 + 5x_2 + 0x_3 = 4\\
1x_1 + 7x_2 + 3x_3 = 20
\end{cases}$$

Resolvendo a primeira linha

$$4x_1 = 12 \implies x_1 = 3$$

Da mesma forma a segunda linha

$$-2(3) + 5x_2 = 4 \implies -6 + 5x_2 = 4 \implies x_2 = 2$$

E por fim a terceira linha

$$1(3) + 7(2) + 3x_3 = 20 \implies 17 + 3x_3 = 20 \implies x_3 = 1$$

Logo temos que $x_1 = 3, x_2 = 2, x_3 = 1$.
\\
\end{enumerate}
\end{exerc}

\begin{exerc}
\begin{enumerate}
\
O código da decomposição LU in-place é
\begin{lstlisting}[language=Python]
import numpy as np

def LU(A):
  m, n = A.shape
  for k in range(n - 1):
    for j in range(k + 1, n):
      aux = A[j, k] / A[k, k]
      A[j, k:n] -= aux * A[k, k:n]
      A[j, k] = aux
\end{lstlisting}

\end{enumerate}
\end{exerc}

\begin{exerc}
\begin{enumerate}[a.]
\
\item
Fazendo o diagrama para calcular as matrizes L e U:

\begin{tabular}{l|l|lll|}
	L & multiplicador &  & A & \\
\hline
	1 &  & 3 & 2 & 4\\
	2 & $m_{21} = \frac{1}{3}$ & 1 & 1 & 2\\
	3 & $m_{31} = \frac{4}{3}$ & 4 & 3 & -2\\
\hline
	4 &  & 0 & $\frac{1}{3}$ & $\frac{2}{3}$\\
	5 & $m_{32} = 1$ & 0 & $\frac{1}{3}$ & $-\frac{8}{3}$\\
\hline
	6 & & 0 & 0 & -8\\
\end{tabular}

Logo a matriz L é:

$$\begin{bmatrix}
	1 & 0 & 0\\
	m_{21} & 1 & 0\\
	m_{31} & m_{32} & 1\\
\end{bmatrix}
=
\begin{bmatrix}
	1 & 0 & 0\\
	\frac{1}{2} & 1 & 0\\
	\frac{4}{3} & 1 & 1\\
\end{bmatrix}
$$

E a matriz U é:

$$\begin{bmatrix}
	3 & 2 & 4\\
	0 & \frac{1}{3} & \frac{2}{3}\\
	0 & 0 & -8\\
\end{bmatrix}
$$

\item
Usando o método da expansão de fatores para calcular a determinante temos que:

$$det(\begin{bmatrix}
	3 & 2 & 4\\
	1 & 1 & 2\\
	4 & 3 & -2\\
\end{bmatrix})
=
(-1)det(\begin{bmatrix}
	2 & 4\\
	3 & -2\\
\end{bmatrix})
+
(1)det(\begin{bmatrix}
	3 & 4\\
	4 & -2\\
\end{bmatrix})
+
(-2)det(\begin{bmatrix}
	3 & 2\\
	4 & 3\\
\end{bmatrix})$$

E como visto que

$$det(\begin{bmatrix}
	2 & 4\\
	3 & -2\\
\end{bmatrix})
= -16$$

$$det(\begin{bmatrix}
	3 & 4\\
	4 & -2\\
\end{bmatrix})
= -22$$

$$det(\begin{bmatrix}
	3 & 2\\
	4 & 3\\
\end{bmatrix})
= 1$$

Temos

$$det(\begin{bmatrix}
	3 & 2 & 4\\
	1 & 1 & 2\\
	4 & 3 & -2\\
\end{bmatrix})
=
(-1)(-16)+(1)(-22)+(-2)(1) = 16 - 22 - 2 = -8$$

\end{enumerate}
\end{exerc}

\newpage

\begin{exerc}
\begin{enumerate}[a.]
\item
Falso. Contra exemplo:

$$C =
\begin{bmatrix}
	1 & 5\\
	4 & 2\\
\end{bmatrix}
,
A = 
\begin{bmatrix}
	0 & 0\\
	0 & 0\\
\end{bmatrix}
$$

Nesse caso $C$ é não singular visto que sua determinante não é nula, ou seja, está nos moldes do problema.

Caso fizermos $C \times A$ temos:

$$C \times A =
\begin{bmatrix}
	0 & 0\\
	0 & 0\\
\end{bmatrix}$$

Cuja determinante é nula, ou seja, $CA$ é singular.

\item
Falso. Visto que C é uma matriz de permutação, sua determinante é $\pm1$. Dado que a determinante do produto entre duas matrizes $A$ e $B$ segue a regra:

$$det(AB) = det(A) det(B)$$

Logo em um caso onde temos o produto entre $C$ e $A$ com $C$ sendo uma matriz de permutação, temos:

$$det(CA) = det(C) det(A)$$

Como $det(C) = \pm1$, existem matrizes no formato de $C$ cuja determinante é $-1$, nesse caso teríamos:

$$det(CA) = (-1) det(A)$$

Logo a afirmação que $det(CA) = det(A)$ é falsa.

\item
Falso. Visto que C é inversível por não ser singular, podemos multiplicar os dois lados do sistema $CAx = Cb$ por $C^{-1}$ (sendo $C^{-1}$ a matriz inversa de $C$) obtendo:

$$C^{-1}CAx = C^{-1}Cb$$

E desenvolvendo utilizando $C^{-1}C = I$ e $AI = A$:

$$Ax = b$$

Assim concluindo que multiplicando cada lado do sistema por uma matriz não singular não altera o resultado já que o sistema pode ser reduzido ao sistema sem a matriz $C$.

\end{enumerate}
\end{exerc}

\end{document}
\grid
\grid