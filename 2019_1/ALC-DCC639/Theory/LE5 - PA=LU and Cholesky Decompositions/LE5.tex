\documentclass[11pt]{article}

\usepackage{amsmath}
\usepackage{amsthm}
\usepackage{amsfonts}
\usepackage{amssymb}
\usepackage[utf8]{inputenc}
\usepackage[T1]{fontenc}
\usepackage[shortlabels]{enumitem}

\usepackage{listings}
\usepackage{color}
%New colors defined below
\definecolor{codegreen}{rgb}{0,0.6,0}
\definecolor{codegray}{rgb}{0.5,0.5,0.5}
\definecolor{codepurple}{rgb}{0.58,0,0.82}
\definecolor{backcolour}{rgb}{0.95,0.95,0.92}

%Code listing style named "mystyle"
\lstdefinestyle{mystyle}{
  backgroundcolor=\color{backcolour},   commentstyle=\color{codegreen},
  keywordstyle=\color{magenta},
  numberstyle=\tiny\color{codegray},
  stringstyle=\color{codepurple},
  basicstyle=\footnotesize,
  breakatwhitespace=false,         
  breaklines=true,                 
  captionpos=b,                    
  keepspaces=true,                 
  numbers=left,                    
  numbersep=5pt,                  
  showspaces=false,                
  showstringspaces=false,
  showtabs=false,                  
  tabsize=2
}

\lstset{style=mystyle}

\topmargin -.5in
\textheight 9in
\oddsidemargin -.25in
\evensidemargin -.25in
\textwidth 7in

\begin{document}

\newtheoremstyle{break}% name
  {}%         Space above, empty = `usual value'
  {}%         Space below
  {}% Body font
  {}%         Indent amount (empty = no indent, \parindent = para indent)
  {\bfseries}% Thm head font
  {.}%        Punctuation after thm head
  {\newline}% Space after thm head: \newline = linebreak
  {}%         Thm head spec

\theoremstyle{definition}
\theoremstyle{break}
\newtheorem{exerc}{Exercício}

\author{Lucas Resende Pellegrinelli Machado (2018126673)}
\title{Algebra Linear Computacional - Lista de Exercicios 5}
\maketitle

\medskip

\begin{exerc}
\begin{enumerate}[a.]
\
\item
Dada a matriz

$$
A = \begin{bmatrix}
	3 & 2 & 4\\
	1 & 1 & 2\\
	4 & 3 & -2\\
\end{bmatrix}
$$

devemos encontrar as matrizes resultantes da decomposição $PA=LU$ da forma.

$$\begin{tabular}{l|l|lll|l|l|}
	L & multiplicador &  & A & & operações  & P\\
\hline
	1 &  & 3 & 2 & 4 & & 1\\
	2 & $m_{21} = \frac{1}{3}$ & 1 & 1 & 2 & & 2\\
	3 & $m_{31} = \frac{4}{3}$ & 4 & 3 & -2 & & 3\\
\hline
	4 &  & 0 & $-\frac{1}{4}$ & $\frac{11}{2}$ & $-\frac{3}{4}l_3 + l_1$ & 1\\
	5 & $m_{32} = 1$ & 0 & $\frac{1}{4}$ & $\frac{5}{2}$ & $-\frac{1}{4}l_3 + l_2$ & 2\\
\hline
	6 & & 0 & 0 & 8 & $l_5 + l_4$ & 1\\
\end{tabular}$$

Pela coluna $P$, é possível concluir que a matriz de permutação será:

$$
P = \begin{bmatrix}
	0 & 0 & 1\\
	0 & 1 & 0\\
	1 & 0 & 0\\
\end{bmatrix}
$$

E pelas outra colunas também temos:

$$
L = \begin{bmatrix}
	1 & 0 & 0\\
	\frac{1}{4} & 1 & 0\\
	\frac{3}{4} & -1 & 1\\
\end{bmatrix}
,
U = \begin{bmatrix}
	4 & 3 & -2\\
	0 & \frac{1}{4} & \frac{5}{2}\\
	0 & 0 & 8\\
\end{bmatrix}
$$

Assim a decomposição fica:

$$
\begin{bmatrix}
	0 & 0 & 1\\
	0 & 1 & 0\\
	1 & 0 & 0\\
\end{bmatrix}
\cdot
\begin{bmatrix}
	3 & 2 & 4\\
	1 & 1 & 2\\
	4 & 3 & -2\\
\end{bmatrix}
=
\begin{bmatrix}
	1 & 0 & 0\\
	\frac{1}{4} & 1 & 0\\
	\frac{3}{4} & -1 & 1\\
\end{bmatrix}
\cdot
\begin{bmatrix}
	4 & 3 & -2\\
	0 & \frac{1}{4} & \frac{5}{2}\\
	0 & 0 & 8\\
\end{bmatrix}
$$
\newpage
\item
Definindo $p$ como o número de permutações entre uma matriz identidade $I$ e a matriz de permutação $P$, sabemos que

$$det(A) = (-1)^p det(U)$$

E como todas as linhas da matriz $P$ são diferentes da matriz $I$ sabemos que o número de trocas é igual a ordem da matriz, no execício, $p = 3$, logo:

$$det(A) = (-1)^3 det(U) = -det(U)$$

E visto que $det(U) = 8$, temos que:

$$det(A) = -8$$

\end{enumerate}
\end{exerc}

\begin{exerc}
\begin{enumerate}
Para calcularmos a solução, primeiros calcularemos a decomposição Cholesky da matriz

$$
A = \begin{bmatrix}
	9 & 18\\
	18 & 52\\
\end{bmatrix}
$$

Utilizaremos das fórmulas (assuma $L$ como a matriz resultante):

$$L_{j, j} = \sqrt{A_{j,j} - \sum_{k = 1}^{j - 1}L_{j, k}^2}$$

$$L_{i, j} = \frac{1}{L_{j, j}}(A_{i, j} - \sum_{k = 1}^{j - 1}L_{i, k}L_{j, k})$$

Logo temos que dada ($L_{1, 2} = 0$ pois $L$ tem que ser triangular inferior)

$$
L = \begin{bmatrix}
	L_{1, 1} & 0\\
	L_{2, 1} & L_{2, 2}\\
\end{bmatrix}
$$

Temos:

$$L_{1, 1} = \sqrt{A_{1, 1} - \sum_{k = 1}^{0}L_{1, k}^2} = \sqrt{9 - 0} = 3$$

$$L_{2, 1} = \frac{1}{L_{1, 1}} (A_{2, 1} - \sum_{k = 1}^{0}L_{2, k}L_{1, k}) = \frac{1}{3}(18 - 0) = 6$$

$$L_{2, 2} = \sqrt{A_{2,2} - \sum_{k = 1}^{1}L_{2, k}^2} = \sqrt{A_{2,2} - L_{2, 1}^2} = \sqrt{52 - 36} = 4$$

Assim

$$
L = \begin{bmatrix}
	3 & 0\\
	6 & 4\\
\end{bmatrix}
$$

Dessa forma temos agora o sistema:

$$L \cdot L^T \cdot x = b$$

Substituindo $L^T \cdot x$ por $y$:

$$L \cdot y = b$$

$$
\begin{bmatrix}
	3 & 0\\
	6 & 4\\
\end{bmatrix}
\cdot
\begin{bmatrix}
	y_1\\
	y_2\\
\end{bmatrix}
=
\begin{bmatrix}
	3\\
	4\\
\end{bmatrix}
$$

Assim, resolvendo o sistema:

$$\begin{cases}
3y_1 + 0y_2 = 3\\
6y_1 + 4y_2 = 4
\end{cases}
\implies
\begin{cases}
y_1 = 1\\
6y_1 + 4y_2 = 4
\end{cases}
\implies
\begin{cases}
y_1 = 3\\
6 + 4y_2 = 4
\end{cases}
\implies
\begin{cases}
y_1 = 1\\
4y_2 = -2
\end{cases}
$$

$$
\implies
y = 
\begin{bmatrix}
	1\\
	-0.5\\
\end{bmatrix}
$$

Retornando à substituição feita:

$$L^T \cdot x = y$$

$$
\begin{bmatrix}
	3 & 6\\
	0 & 4\\
\end{bmatrix}
\cdot
\begin{bmatrix}
	x_1\\
	x_2\\
\end{bmatrix}
=
\begin{bmatrix}
	1\\
	-0.5\\
\end{bmatrix}
$$

Resolvendo:

$$
\begin{cases}
3x_1 + 6x_2 = 1\\
0x_1 + 4x_2 = -0.5
\end{cases}
\implies
\begin{cases}
3x_1 + 6x_2 = 1\\
x_2 = -0.125
\end{cases}
\implies
\begin{cases}
3x_1 - 0.75 = 1\\
x_2 = -0.125
\end{cases}
\implies
\begin{cases}
3x_1 = 1.75\\
x_2 = -0.125
\end{cases}
$$

$$
\implies
x = 
\begin{bmatrix}
	\frac{7}{12}\\
	-\frac{1}{8}\\
\end{bmatrix}
$$

\end{enumerate}
\end{exerc}

\newpage

\begin{exerc}
\begin{enumerate}[a.]
\
\item

Dado que $x^{(0)} =
\begin{bmatrix}
	0.5\\
	-0.1\\
\end{bmatrix}$, temos que:

$$Ax^{(0)} =
\begin{bmatrix}
	9 & 18\\
	18 & 52\\
\end{bmatrix}
\cdot
\begin{bmatrix}
	0.5\\
	-0.1\\
\end{bmatrix}
=
\begin{bmatrix}
	2.7\\
	3.8\\
\end{bmatrix}
$$

Calculando o erro residual $r^{(0)}$:

$$r^{(0)} = b - Ax^{(0)} =
\begin{bmatrix}
	3\\
	4\\
\end{bmatrix}
-
\begin{bmatrix}
	2.7\\
	3.8\\
\end{bmatrix}
=
\begin{bmatrix}
	0.3\\
	0.2\\
\end{bmatrix}
$$

\item
Para refinar $x{(0)}$ e obter a solução refinada $x_r$ temos que:

$$x_r = x^{(0)} + c^{(0)}$$

Para achar $c^{(0)}$ precisamos de

$$Ly = r^{(0)} \text{ e } L^Tc^{(0)} = y$$

Assim

$$\begin{bmatrix}
	3 & 0\\
	6 & 4\\
\end{bmatrix}
\cdot
\begin{bmatrix}
	y_1\\
	y_2\\
\end{bmatrix}
=
\begin{bmatrix}
	0.3\\
	0.2\\
\end{bmatrix}
\implies
\begin{cases}
3y_1 + 0y_2 = 0.3\\
6y_1 + 4y_2 = 0.2
\end{cases}
\implies
y
=
\begin{bmatrix}
	0.1\\
	-0.1\\
\end{bmatrix}
$$

E então:

$$
\begin{bmatrix}
	3 & 6\\
	0 & 4\\
\end{bmatrix}
\begin{bmatrix}
	c^{(0)}_1\\
	c^{(0)}_2\\
\end{bmatrix}
=
\begin{bmatrix}
	0.1\\
	-0.1\\
\end{bmatrix}
\implies
\begin{cases}
3c^{(0)}_1 + 6c^{(0)}_2 = 0.1\\
0c^{(0)}_1 + 4c^{(0)}_2 = -0.1
\end{cases}
\implies
c^{(0)}
=
\begin{bmatrix}
	\frac{1}{12}\\
	-\frac{1}{40}\\
\end{bmatrix}
$$

Com o valor de $c^{(0)}$ temos agora:

$$x_r =
\begin{bmatrix}
	\frac{1}{2}\\
	-\frac{1}{10}\\
\end{bmatrix}
+
\begin{bmatrix}
	\frac{1}{12}\\
	-\frac{1}{40}\\
\end{bmatrix}
=
\begin{bmatrix}
	\frac{7}{12}\\
	-\frac{1}{8}\\
\end{bmatrix}$$

\end{enumerate}
\end{exerc}

\begin{exerc}
\begin{enumerate}[a.]
\
\item Verdadeiro. Visto que $AA^T$ é uma matriz simética definida positiva, ela pode ser decomposta pela decomposição Cholesky.

\item Falso. Como a matriz $A$ tem posto $n$, é impossível reconstrui-la por meio de uma multiplicação de matrizes de posto menor que $n$. Como a matriz $U$ não tem posto $n$ visto que uma de suas linhas é nula e pode ser representada por qualquer outra linha multiplicada por 0, então a afirmativa é falsa;

\item Verdadeiro. Visto que Cholesky precisa resolver apenas uma matriz, ela precisa de metade do tempo que o LU já que a LU gera duas matrizes. Porém em questão de espaço, as duas precisam de gerar matrizes intermediarias para o processamento, ou seja, ocupam o mesmo espaço.

\end{enumerate}
\end{exerc}

\end{document}
\grid
\grid