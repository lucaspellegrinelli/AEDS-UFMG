\documentclass[11pt]{article}

\usepackage{amsmath}
\usepackage{amsthm}
\usepackage{amsfonts}
\usepackage{amssymb}
\usepackage[utf8]{inputenc}
\usepackage[T1]{fontenc}
\usepackage[shortlabels]{enumitem}

\usepackage{listings}
\usepackage{color}
%New colors defined below
\definecolor{codegreen}{rgb}{0,0.6,0}
\definecolor{codegray}{rgb}{0.5,0.5,0.5}
\definecolor{codepurple}{rgb}{0.58,0,0.82}
\definecolor{backcolour}{rgb}{0.95,0.95,0.92}

%Code listing style named "mystyle"
\lstdefinestyle{mystyle}{
  backgroundcolor=\color{backcolour},   commentstyle=\color{codegreen},
  keywordstyle=\color{magenta},
  numberstyle=\tiny\color{codegray},
  stringstyle=\color{codepurple},
  basicstyle=\footnotesize,
  breakatwhitespace=false,         
  breaklines=true,                 
  captionpos=b,                    
  keepspaces=true,                 
  numbers=left,                    
  numbersep=5pt,                  
  showspaces=false,                
  showstringspaces=false,
  showtabs=false,                  
  tabsize=2
}

\lstset{style=mystyle}

\topmargin -.5in
\textheight 9in
\oddsidemargin -.25in
\evensidemargin -.25in
\textwidth 7in

\begin{document}

\newtheoremstyle{break}% name
  {}%         Space above, empty = `usual value'
  {}%         Space below
  {}% Body font
  {}%         Indent amount (empty = no indent, \parindent = para indent)
  {\bfseries}% Thm head font
  {.}%        Punctuation after thm head
  {\newline}% Space after thm head: \newline = linebreak
  {}%         Thm head spec

\theoremstyle{definition}
\theoremstyle{break}
\newtheorem{exerc}{Exercício}

\author{Lucas Resende Pellegrinelli Machado (2018126673)}
\title{Algebra Linear Computacional - Lista de Exercicios 1}
\maketitle

\medskip

\begin{exerc}
\begin{enumerate}
Visto que a matriz que dita a transição entre os estados depois de 1 ano é
\[
E_{+1} =
\begin{bmatrix}
    0.90 & 0.07 & 0.02 & 0.01 \\
    0.00 & 0.93 & 0.05 & 0.02 \\
    0.00 & 0.00 & 0.85 & 0.15 \\
    0.00 & 0.00 & 0.00 & 1.00
\end{bmatrix}
\]

e temos que o estado atual da população é

\[
E_{atual} = 
\begin{bmatrix}
    0.85 & 0.10 & 0.05 & 0.00
\end{bmatrix}
\]

Para calcular o estado da população daqui a 10 anos, temos apenas que calcular

$$E_{10 anos} = E_{atual} \times (E_{+1})^{10}$$

Que com Python e a biblioteca Numpy pode ser calculada usando:

\begin{lstlisting}[language=Python]
import numpy as np

# Matriz de transicao
a = np.array([[0.9, 0.07, 0.02, 0.01],
              [0.0, 0.93, 0.05, 0.02],
              [0.0, 0.0,  0.85, 0.15],
              [0.0, 0.0,  0.0,  1.0]])

# Configuracao atual da populacao
b = np.array([0.85, 0.1, 0.05, 0.0])

# Calcula b x a^10
r = b @ np.linalg.matrix_power(a, 10)
print(r)
\end{lstlisting}

Que nos dá como resposta

\[
E_{10anos} = 
\begin{bmatrix}
    0.29637667 & 0.3167509 & 0.1342175 & 0.25265492
\end{bmatrix}
\]

Logo temos que a população após 10 anos é composta por aproximadamente 29.6\% assintomáticos, 31.7\% sintomáticos, 13.4\% com AIDS e 25.3\% mortos.

\end{enumerate}
\end{exerc}

\newpage

\begin{exerc}
\begin{enumerate}[a.]
De acordo com o enunciado, o número tem 5 bits de mantissa e expoentes estão no intervalo $[6, -7] \in \mathbb{N}$.

\item 
O número tem que ser positivo, logo o primeiro bit tem de ser 0 para isso. Agora, temos que escolher também a menor combinação entre mantissa e expoente possível. Como os expoentes estão no intervalo $[6, -7] \in \mathbb{N}$, para que o número seja o menor possível, precisamos da menor mantissa possível elevada à -6. Para a menor mantissa, temos apenas que colocar os 5 bits como 0. Dessa forma temos:

$$Primeiro bit = 0 = (-1)^0$$
$$Mantissa = (1.00000)_2$$
$$Expoente = -6$$

Assim:

$$(-1)^0 \cdot (1.00000)_{2} \cdot 2^{-6} = 2^{-6}$$



\item
O problema é bem similar com a letra (a). As únicas diferenças é que queremos a maior mantissa possível elevada ao maior expoente possível. Como os expoentes estão no intervalo $[6, -7] \in \mathbb{N}$, o maior possível é 7. Para a mantissa é só colocar os 5 bits como 1. Assim temos:

$$Primeiro bit = 0 = (-1)^0$$
$$Mantissa = (1.11111)_2$$
$$Expoente = 7$$

Assim:

$$(-1)^0 \cdot (1.11111)_2 \cdot 2^7 \approx 2^8$$

\item
Da mesma forma que transformamos números em notação científica em base 10 (no estilo $x \times 10^y$) para números explicitos movendo as vírgulas para a direita até que exista um número 1 à esquerda da virgula, faremos o mesmo com números em base 2.

$$2^3 \cdot (0.01010)_2 \implies 2^2 \cdot (0.10100)_2 \implies 2^1 \cdot (1.01000)$$

Assim, depois de adicionar o primeiro bit que será 0 visto que o número é positivo, temos que a resposta é:

$$(-1)^0 \cdot (1.01000)_2 \cdot 2^1$$

\item
Esse exercício é analogo ao anterior, porém como se trata de expoentes negativos, movemos a virgula para a esquerda até que só exista um dígito à esquerda da virgula. Outra diferença é que o número dessa vez é negativo, logo o primeiro bit será 1.

$$-2^{-2} \cdot (1101.01010)_2 \implies -2^{-3} \cdot (110.10101)_2 \implies -2^{-4} \cdot (11.01010)_2 \implies -2^{-5} \cdot (1.10101)_2$$

Um detalhe importante é que a medida que fazemos isso, perdemos informações de bits que são empurrados além da capacidade de armazenamento para a direita.

Mas direto à resposta, adicionando o bit de sinal:

$$(-1)^1 \cdot 2^{-5} \cdot (1.10101)_2$$

\item
Como o número é maior que 1, ele é positivo, logo o primeiro bit é 0. Para garantir que o número será maior que 1, diremos que o expoente é 0 e que a mantissa tem que ser maior que 0. Dessa forma, quando fizermos $expoente \cdot mantissa$ teremos um resultado maior que 1 (visto que com o $expoente = 0$, teremos $2^0 = 1$).

O desafio agora é encontrar a menor mantissa possível maior que $(1.00000)_2$, que levando em conta que os digitos significativos representam números maiores a medida que você olha bits mais à esquerda, se torna fácil perceber que a menor mantissa possível maior que essa é uma com um 1 no bit menos significativo, ou seja, $(1.00001)_2$.

Dessa forma, com o sinal sendo positivo, expoente sendo 0 e a mantissa sendo $(1.00001)_2$, temos que a resposta é:

$$(-1)^0 \cdot (1.00001)_2 \cdot 2^0 \approx 1 + 2^{-5}$$

\item
Como o $\epsilon_{machine}$ é a maior precisão que nossa máquina consegue atingir, um jeito fácil de verificar qual é esse valor é ver qual a metade da diferença entre 1 e o menor número possível de se representar na máquina maior que 1.

Na questão (e) descobrimos que tal número é $(-1)^0 \cdot (1.00001)_2 \cdot 2^0 \approx 1 + 2^{-5}$, logo a diferença para 1 é $2^{-5}$ e a metade disso é $2^{-6}$, sendo essa a resposta.

\end{enumerate}
\end{exerc}

\begin{exerc}
\begin{enumerate}

Primeiro temos que

$$A_{m \times n} = U_{m \times m} \Sigma_{m \times m} V^T_{n \times n}$$

Logo, usando a propriedade que $(A B)^T = B^T A^T$

$$A^T_{n \times m} = V_{n \times n} \Sigma^T_{m \times m} U^T_{m \times m}$$

\item
Temos que:

$$A A^T = U \Sigma V^T V \Sigma^T U^T$$

Usando que como $V$ é ortogonal, $V^T V = I$

$$A A^T = U \Sigma \Sigma^T U^T$$

Usando que $\Sigma$ é uma matriz diagonal retangular, temos que $\Sigma \Sigma^T = \Sigma^2$

$$A A^T = U \Sigma^2 U^T$$

\item
Da mesma forma que:

$$A^T A = V \Sigma^T U^T U \Sigma V^T$$

Usando que como $U$ é ortogonal, $U^T U = I$

$$A^T A = V \Sigma^T \Sigma V^T$$

Usando que $\Sigma$ é uma matriz diagonal retangular, temos que $\Sigma^T \Sigma = \Sigma^2$

$$A A^T = V \Sigma^2 V^T$$

\end{enumerate}
\end{exerc}

\begin{exerc}
\begin{enumerate}
Primeiro temos que mostrar que $U$ e $V$ são ortonormais, ou seja, todas as colunas são unitárias e o produto escalar entre todas é 0.

- Testando se $U$ é ortonormal:
\[
C_1 = 
\begin{bmatrix}
    -0.632 & 0.316 & -0.316 & 0.632
\end{bmatrix}
\]
\[
C_2 =
\begin{bmatrix}
    0.000 & -0.707 & -0.707 & 0.000
\end{bmatrix}
\]

$$C_1 \cdot C_2 = (0.316)(-0.707) + (-0.316)(-0.707) = 0$$
\[
|C_1| = 
\sqrt{ (-0.632)^2 + (0.316)^2 + (-0.316)^2 + (0.632)^2}
= 1
\]
\[
|C_2| = 
\sqrt{ (0.000)^2 + (-0.707)^2 + (-0.707)^2 + (0.000)^2}
= 1
\]\\
Como $C_1 \cdot C_2 = 0$, $|C_1| = 1$ e $|C_2| = 1$, então $U$ é ortonormal.
\\\\
- Testando se $V$ é ortonormal:
\[
C_1 = 
\begin{bmatrix}
    -0.707 & 0.707
\end{bmatrix}
\]
\[
C_2 =
\begin{bmatrix}
    -0.707 & -0.707
\end{bmatrix}
\]

$$C_1 \cdot C_2 = (-0.707)(-0.707) + (0.707)(-0.707) = 0$$
\[
|C_1| = 
\sqrt{ (-0.707)^2 + (0.707)^2}
= 1
\]
\[
|C_2| = 
\sqrt{ (-0.707)^2 + (-0.707)^2 }
= 1
\]\\
Como $C_1 \cdot C_2 = 0$, $|C_1| = 1$ e $|C_2| = 1$, então $V$ é ortonormal.

\newpage


O que resta mostrar é que de fato

$$C \approx U \Sigma V^T$$

O que é facil mostrar com o seguinte script em Python

\begin{lstlisting}[language=Python]
import numpy as np

U = np.array([[-0.632, 0.000],
			[0.316, -0.707],
			[-0.316, -0.707],
 			[0.632, 0.0]])
              
Sig = np.array([[2.236, 0.0],
			[0.0, 1.0]])
				
Vt = np.array([[-0.707, 0.707],
			[-0.707, -0.707]])
			   
C = U @ Sig @ Vt
print(C)
\end{lstlisting}

Que gera o resultado (valores arredendados para 4 casas decimais):

\[
\begin{bmatrix}
    0.9991 & -0.9991 \\
    0.0003 & 0.9994 \\
    0.9994 & 0.0003 \\
    -0.9994 & 0.9994
\end{bmatrix}
\approx
\begin{bmatrix}
    1 & -1 \\
    0 & 1 \\
    1 & 0 \\
    -1 & 1
\end{bmatrix}
\]

O que é correto visto que a representação em SVD de uma matriz será uma aproximação da mesma, logo com os valores com 3 casas decimais indicados pelo exercício temos um pequeno erro em relação a matriz desejada (na casa de $10^{-4}$).

\end{enumerate}
\end{exerc}

\begin{exerc}
\begin{enumerate}
Temos que achar um vetor $v = (x, y)$ tal que com

\[
C^* = 
\begin{bmatrix}
    1 & -1 \\
    0 & 1 \\
    1 & 0 \\
    -1 & 1 \\
    1 & 1
\end{bmatrix}
\text{ e }
U^* = 
\begin{bmatrix}
    -0.632 & 0.000 \\
    0.316 & -0.707 \\
    -0.316 & -0.707 \\
    0.632 & 0.000 \\
    x & y
\end{bmatrix}
\]

Temos

$$C^* = U^* \Sigma V^T$$

Resolvendo $U^* \times \Sigma$:
\[
U^* \times \Sigma = 
\begin{bmatrix}
	(-0.632 \cdot 2.236 + 0.0 \cdot 0.0) & (-0.632 \cdot 0.0 + 0.0 \cdot 1.0) \\
	(0.316 \cdot 2.236 - 0.707 \cdot 0.0) & (0.316 \cdot 0.0 - 0.707 \cdot 1.0) \\
	(-0.316 \cdot 2.236 - 0.707 \cdot 0.0) & (-0.316 \cdot 0.0 - 0.707 \cdot 1.0) \\
	(0.632 \cdot 2.236 + 0.0 \cdot 0.0) & (0.632 \cdot 0.0 + 0.0 \cdot 1.0) \\
	(x \cdot 2.236 + y \cdot 0.0) & (x \cdot 0.0 + y \cdot 1.0)
\end{bmatrix}
=
\begin{bmatrix}
	-1.4131 & 0 \\
	0.7065 & -0.707 \\
	-0.7065 & -0.707 \\
	1.4131 & 0 \\
	x \cdot 2.236 & y
\end{bmatrix}
\]

E agora resolvendo $(U^* \times \Sigma) \times V^T$

\[
(U^* \times \Sigma) \times V^T = 
\begin{bmatrix}
	-1.4131 \cdot -0.707 + 0.0 \cdot -0.707 & -1.4131 \cdot 0.707 + 0.0 \cdot -0.707 \\
	0.7065 \cdot -0.707 + -0.707 \cdot -0.707 & 0.7065 \cdot 0.707 - 0.707 \cdot -0.707 \\
	-0.7065 \cdot -0.707 + -0.707 \cdot -0.707 & -0.7065 \cdot 0.707 - 0.707 \cdot -0.707 \\
	1.4131 \cdot -0.707 + 0 \cdot -0.707 & 1.4131 \cdot 0.707 + 0.0 \cdot -0.707 \\
	x \cdot 2.236 \cdot -0.707 + y \cdot -0.707 & x \cdot 2.236 \cdot 0.707 + y \cdot -0.707 
\end{bmatrix}
=
\]
\[
=
\begin{bmatrix}
	0.9991 & -0.9991 \\
	-0.4994 + 0.4998 & 0.4994 + 0.4998 \\
	0.4994 + 0.4999 & -0.4994 + 0.4998 \\
	-0.9991 & 0.9991 \\
	-1.5808 x - 0.707 y & 1.5808 x - 0.707 y
\end{bmatrix}
=
\begin{bmatrix}
	0.9991 & -0.9991 \\
	0.0004 & 0.9992 \\
	0.9992 & 0.0004 \\
	-0.9991 & 0.9991 \\
	-1.5808 x - 0.707 y & 1.5808 x - 0.707 y
\end{bmatrix}
\]

Assim, voltando ao problema que o novo usuário deu nota positiva aos dois sites, temos que:

\[
\begin{bmatrix}
	-1.5808 x - 0.707 y & 1.5808 x - 0.707 y
\end{bmatrix}
=
\begin{bmatrix}
	1 & 1
\end{bmatrix}
\]

Ou seja, temos o sistema:
$$\begin{cases} -1.5808 x - 0.707 y = 1 \\ 1.5808 x - 0.707 y = 1 \end{cases}$$

E resolvendo o sistema temos

$$\begin{cases} x = 0 \\ y = -1.4144 \end{cases}$$

Logo temos que:

\[
U^* = 
\begin{bmatrix}
    -0.632 & 0.000 \\
    0.316 & -0.707 \\
    -0.316 & -0.707 \\
    0.632 & 0.000 \\
    0 & -1.4144
\end{bmatrix}
\]

E testando a nova matriz na fórmula $C^* = U^* \Sigma V^T$, temos que:

\[
\begin{bmatrix}
    0.9991 & -0.9991 \\
    0.0003 & 0.9994 \\
    0.9994 & 0.0003 \\
    -0.9994 & 0.9994 \\
    0.9999 & 0.9999
\end{bmatrix}
\approx
\begin{bmatrix}
    1 & -1 \\
    0 & 1 \\
    1 & 0 \\
    -1 & 1 \\
    1 & 1
\end{bmatrix}
\]

Exatamente como queriamos

\end{enumerate}
\end{exerc}

\end{document}
\grid
\grid